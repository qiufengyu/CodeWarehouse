\documentclass[11pt]{article}
\usepackage[BoldFont,SlantFont,CJKchecksingle]{xeCJK}
\usepackage{enumerate}
\usepackage{multirow}
\usepackage{amsmath}
\usepackage[colorlinks=true,pdfborder=000,citecolor=magenta,linkcolor=black]{hyperref}
\usepackage[left=1in,right=1in,top=1in,bottom=1in]{geometry}
\linespread{1.4}
\parindent=2em
\setmainfont{Times New Roman}
\setCJKmainfont[BoldFont=SimHei]{SimSun} %
\setCJKmonofont{SimSun}% 设置缺省中文字体
\begin{document}
\begin{flushleft}
{\Huge{\textbf{NLP\ \ Sentiment Classification\ \ 实验报告}}}\\
\end{flushleft}
\section{任务描述}
\hspace{1.6em} 情感二元分类。根据一段用户主观描述性文本,进行分析和处理,预测或推测该文本的情感取向是具有~\underline{P}ositive或是~\underline{N}egative的色彩。\\
\indent 任务过程主要有:
\begin{enumerate}
  \item 根据训练数据集训练预测模型:对于文本的处理包括分词、特征提取等,建立并根据特征值,训练朴素贝叶斯模型(Naïve Bayes Model)
  \item 根据预测模型进行预测:以最大概率预测某文本的情感取向为$+$1(P)或$-$1(N)。
  \item 预测结果评价:使用多种评估标准,如Presicion、Recall、F1-measure和Accuracy,体现预测结果的优劣。
\end{enumerate}
\section{分类算法}
\hspace{1.6em} 主要采用了基于朴素贝叶斯算法的模型,根据文本表示、特征选择的细节上,本实验中采用了Bernoulli文本模型和Multinomial文本模型两种。\\
\\
\begin{tabular}{|c|c|p{8cm}|}
\hline
具体模型 & 相同点 & \multicolumn{1}{c|}{不同点} \\
\hline
\multirow{3}{*}{Bernoulli 文本模型}& \multirow{6}{3cm}{都是基于朴素贝叶斯文本模型,给出最大概率的情感分类取向。}& 文本表示:使用一个二元值\{\ 0, 1\ \}列向量\\ \cline{3-3}
& & \multirow{2}{*}{概率模型:$\hat{P}(w_t|c_k)=\dfrac{n_k(w_t)}{N_k}$}\\
 & & \\ \cline{1-1} \cline{3-3}
\multirow{3}{*}{Multinomial 文本模型}& & 文本表示:使用一个以频数为数值的列向量 \\ \cline{3-3}
& & \multirow{2}{*}{概率模型:$\hat{P}(w_t|c_k)=\dfrac{\sum_{i=1}^{N}D_{it}z_{ik}}{\sum_{s=1}^{|V|}\sum_{i=1}^{N}D_{is}z_{ik}}$}\\
& & \\
\hline
\end{tabular}
\section{文本表示}
\hspace{1.6em} 首先都对文本进行分词处理,使用了IKAnalyzer工具包\footnote{Google Code 开源项目:http://code.google.com/p/ik-analyzer}。这是一个java环境下常用的分词工具\footnote{使用手册:http://wenku.baidu.com/view/f27d63d676eeaeaad1f33080.html}进行文本分词。根据不同的模型,来表示每一个训练文档。
\subsection{Bernoulli 文本模型}\label{bernoulli}
\hspace{1.6em} 每一段文本用一个分量只有二元取值的列向量来表示。用0表示对应的单词在该文本中未出现,1表示对应的单词在文本中出现。用$\mathbf{V}$表示词汇表,每一个文本都可以表示成$|\mathbf{V}|$维的0-1列向量,但是我们没有考虑每个单词出现的次数这一因素。
\subsection{Multinomial 文本模型}
\hspace{1.6em} 每个文本都用一个分量由整数值组成的列向量,每个值代表对应的词出现的次数(频数)。在文本的表示中,利用了包括频数在内的信息,相比[\ref{bernoulli}]具有更多的有效信息,但开销更大。
\section{特征选择}
\subsection{Bernoulli 文本模型}\label{bernoulli5}
\hspace{1.6em} 我们用$D_i$表示第$i$个文本的特征向量,$D_{it}=0$表示单词$w_t$并没有出现(absence)在文本$D_i$中,$D_{it}=1$表示单词$w_t$出现(presence)在文本$D_i$中;则$P(w_t|c_k)$则表示单词$w_t$出现在$c_k$类型文本中的概率,$(1-P(w_t|c_k))$反之。那么,文本特征向量$D_i$中,对于单词$w_t$的因素,使其情感属于某一类$c_k$的概率,即已知是属于$c_k$类,那么这个单词则是第$i$个文本中的单词$w_t$的概率是:
\begin{equation}\label{bernoulli1}
P(D_{it}|c_k)=D_{it}P(w_t|c_k)+(1-D_{it})(1-P(w_t|c_k))
\end{equation}
\indent 同时我们对于每个单词$w_t \in \mathbf{V}$,利用(\ref{bernoulli1})式,由于我们已经假设每个单词的出现都是独立的,对概率值进行连乘即可得到已知为$c_k$类的情感,是文本$D_i$的概率为:
\begin{equation}\label{bernoulli2}
P(D_{i}|c_k)=\prod_{t=1}^{|\mathbf{V}|}P(D_{it}|c_k)\\
=\prod_{t=1}^{|\mathbf{V}|}[D_{it}P(w_t|c_k)+(1-D_{it})(1-P(w_t|c_k))]
\end{equation}
\indent 通过训练文本,我们可以得出$P(w_t|c_k)$的估计值($n_k(w_t)$表示在$c_k$类情感的文本中单词$w_t$出现的次数,$N_k$表示所有的训练文本中所有属于$c_k$类情感的数量):
\begin{equation}\label{bernoulli3}
\hat{P}(w_t|c_k) = \frac{n_k(w_t)}{N_k}
\end{equation}
\indent 那么对于某个测试文本$D_j$,我们的目标是找到$c_k$,使得$P(c_k|D_j)$的概率值最大,就可以判定为属于$c_k$类的情感文档。运用贝叶斯公式和全概率公式,我们可以对求$P(c_k|D_j)$最大值进行转化,即求$P(D_j|c_k)\cdot P(c_k)$的最大值。先转化,再利用(\ref{bernoulli2})式展开,可以得到如下目标式:
\begin{equation}\label{bernoulli4}
\begin{split}
{\arg\!\max}_{c_k} P(c_{k}|D_j)&={\arg\!\max}_{c_k}P(D_j|c_k)\cdot P(c_k)\\
&={\arg\!\max}_{c_k}P(c_k)\prod_{t=1}^{|\mathbf{V}|}[D_{jt}P(w_t|c_k)+(1-D_{jt})(1-P(w_t|c_k))]
\end{split}
\end{equation}
\indent 实际情况下,$\forall k_1, k_2, P(c_{k_1}) = P(c_{k_2})$,因此$P(c_k)$部分可以不计算。\\
\indent 由于可能会出现0值,所以采用了Laplace校准\footnote{http://hijiangtao.github.io/2014/09/08/Bayes-theorem/};并且,概率值通常来说较小,要进行乘法计算,最终的结果往往很小,与0十分接近,大小比较不明显,所以将其取自然对数,最终相加同样取满足条件的最大的$c_k$,如果仍然因为太过接近而无法判断,则随机给出分类,最终给出属于P类情感或N类情感的结果。
\subsection{Multinomial 文本模型}
\hspace{1.6em} Multinomial文本模型在文本特征表示上,主要利用了单词出现频数的信息,因此,文本的特征向量$D_{it}$中的分量表示在文本$D_i$中单词$w_t$出现的次数。类似地,如果已知情感态度属于$c_k$类,那么是文本$D_i$的概率是(其中,$n_i$表示文本$D_i$的总词数):
\begin{equation}\label{multi1}
P(D_{i}|c_k)=\frac{n_i!}{\prod_{t=1}^{|\mathbf{V}|}D_{it}!}\prod_{t=1}^{|\mathbf{V}|} P(w_t|c_k)^{D_{it}}
\end{equation}
\indent 在Multinomial文本模型下,我们规定$z_{ik}=1$表示文本$D_i\in c_k$,否则$z_{ik}=0$,$P(w_t|c_k)$的估计值采用如下的方式计算:
\begin{equation}\label{multi2}
\hat{P}(w_t|c_k)=\frac{\sum_{i=1}^{N}D_{it}z_{ik}}{\sum_{s=1}^{|V|}\sum_{i=1}^{N}D_{is}z_{ik}}
\end{equation}
\indent 与(\ref{bernoulli4})类似,我们最终的目标函数可以化简为:
\begin{equation}
{\arg\!\max}_{c_k} P(c_{k}|D_j)={\arg\!\max}_{c_k}P(c_k)\prod_{h=1}^{\text{len}(D_i)}P(u_h|c_k)
\end{equation}
\indent 其中,$u_h$为词汇表中出现在文本$D_j$中的单词。并且使用了与[\ref{bernoulli5}]最后所述相同的技巧。
\section{结果分析}
\hspace{1.6em} 在本次实验中,主要采用了四种的评价标准:Precision、Recall、F1-Measure、Accuracy。我们有四个计数:
\begin{enumerate}
  \item TP:实际为Positive,预测为Positive
  \item FP:实际为Negative,预测为Positive
  \item TN:实际为Negative,预测为Negative
  \item FN:实际为Positive,预测为Negative
\end{enumerate}
\hspace{1.6em} 四个评价标准分别定义为:
\begin{enumerate}
  \item Precision = $\dfrac{\text{TP}}{\text{TP}+\text{FP}}$
  \item Recall = $\dfrac{\text{TP}}{\text{TP}+\text{FN}}$
  \item F1 = $\dfrac{2 \cdot \text{Precision} \cdot \text{Recall}}{\text{Precision}+\text{Recall}}$
  \item Accuracy = $\dfrac{\text{TP}+\text{TN}}{\text{TP}+\text{FP}+\text{TN}+\text{FN}}$
\end{enumerate}
\hspace{1.6em} 最终两种模型的结果如下:
\begin{center}
\begin{tabular}{|c|c|c|c|c|}
  \hline
  具体模型 & Precision & Recall & F1 & Accuracy \\ \hline
  Bernoulli 文本模型 & 51.6528\% & 89.8500\% & 65.5959\% & 52.8750\% \\ \hline
  Multinomial 文本模型 & 84.8534\% & 52.1000\% & 64.5601\% & 71.4000\% \\
  \hline
\end{tabular}
\end{center}
\hspace{1.6em} Multinomial模型由于利用了更多的数据,基本上结果要优于Bernoulli模型。在执行效率上,由于Multinomial模型进行了一些局部优化,故耗时在15seconds左右,Bernoulli模型耗时需要60seconds左右。
\section{运行环境}
\hspace{1.6em} 机器环境:Windows 7 64-bit, Core i3 @2.30GHz, RAM: 6GB;JDK版本:1.8.0\_60,with Eclipse Mars \\
\indent 包括模型训练和模型测试两部分,生成了两个jar可执行文件,可直接双击运行,或在对应目录下命令行中使用\\
\indent \texttt{java -jar *.jar},*为对应文件名,请保证对应的数据文件在对应目录的data目录中。其他未尽事宜请联系作者或查阅ReadMe.txt。
\end{document} 