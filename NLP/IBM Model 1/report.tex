\documentclass[11pt]{article}
\usepackage[BoldFont,SlantFont,CJKchecksingle]{xeCJK}
\usepackage{enumerate}
\usepackage{multirow}
\usepackage{amsmath}
\usepackage{amssymb}
\usepackage{paralist}
\usepackage{algorithm}
\usepackage{algorithmicx}
\usepackage[noend]{algpseudocode}
\usepackage{footmisc}
\usepackage[colorlinks=true,pdfborder=000,citecolor=magenta,linkcolor=black]{hyperref}
\usepackage[left=1in,right=1in,top=1in,bottom=1in]{geometry}
\linespread{1.4}
\parindent=2em
\setmainfont{Times New Roman}
\setCJKmainfont[BoldFont=SimHei]{SimSun} %
\setCJKmonofont{SimSun}% 设置缺省中文字体

\begin{document}
\begin{flushleft}
{\Huge{\textbf{NLP\ \ IBM语言翻译模型\ \ 实验报告}}}\\
\end{flushleft}
\section{任务描述}
\hspace{1.6em} IBM模型是统计机器翻译模型的基础,也是现在统计机器翻译中主流技术中的重要一步~[\ref{note1}]。IBM Model 1主要是考虑了词的对应关系,引入了“对齐”(Alignment)的概念;相比之下,Model 2还考虑了词的位置变化概率这一因素,显然要较Model 1更进化。\\
\indent 输入:英文句子和对应的中文句子,输出:词对齐结果,由元组$(c_j-e_i)$,即中文词对应的英文词在句中的位置(从0开始标定)

\section{IBM Model 1}
\subsection{模型建立}
\hspace{1.6em} 定义英文句子的集合为$\mathcal{E}$,中文句子的集合为$\mathcal{C}$,英文句子$E=(e_1, e_2, \cdots, e_l)$,长度为$l$,中文句子为$C=(c_1, c_2, \cdots, c_m)$,长度为$m$。在此基础上定义对齐$a$,是一个长为$m$的数组,$a[j]=i$表示对应关系:$c_j\rightarrow e_i$,在IBM Model 1中翻译概率可以等价转化为~[\ref{note3}]
\begin{equation}\label{transp1}
P(C,a|E)=P(a|E,l,m)\times P(C|a,E,m,l)=\frac{\epsilon}{(l+1)^m}\times \prod_{j=1}^{m}t(c_j|e_{a[j]})
\end{equation}
\hspace{1.6em} 其中,$\epsilon$是一个标准化常数;通常我们会在英文句子$E$中增加一个"NULL"作为$e_0$,所以是$(l+1)$;$t(c|e)$表示从英语词$e$生成(翻译)汉语词$c$的条件概率。\\
\indent IBM Model 1有一个很强的假设,就是所有可能的对齐概率都是一样的,即
\begin{equation}\label{model1}
P(a|E,m)=\frac{\epsilon}{(l+1)^m}
\end{equation}
\hspace{1.6em} 这在Model 2中做了改进,增加了位置关系概率~[\ref{note2}]~[\ref{note4}]
\begin{equation}\label{model2q}
P(a|E,m) = \prod_{j=1}^{m}q(a[j]|j, l, m)
\end{equation}
\hspace{1.6em} 其中概率$q(j|i, l, m)$表示在给定英文和中文句长为$l$和$m$时,对齐$a[i]=j$的条件概率。事实上,$q(j|i, l, m)$定义的是关于所有可能的$(i,l,m)$的一系列概率分布:$\langle q(0|i, l, m), q(1|i, l, m), \cdots , q(l|i, l, m)\rangle$,这对于英文句子$E$来说,仍然有一个独立的假设。综上,IBM Model 2的翻译概率即为
\begin{equation}\label{transp2}
P(C,a|E)=\prod_{i=1}^{m}q(a[i]|i, l, m)\times \prod_{j=1}^{m}t(c_j|e_{a[j]})
\end{equation}
\subsection{模型实现}
\hspace{1.6em} 假设当前没有Alignment的方式,需要自己训练,就需要用到EM算法进行迭代~[\ref{note3}]。首先对需要计算的参数$t(c|e)$进行初始化,一般采用标准化的方法,即为$1/|\mathcal{C}|$,$|\mathcal{C}|$表示所有的中文词个数。\\
\indent E-step:使用当前状态的模型,计算概率,$t(c|e)=count(c|e)/total(e)$进行赋值;\\
\indent M-step:通过上一步得到的$t(c|e)$,重新进行计数,以新的概率分布为权重,更新$count(c|e)$和$total(e)$,伪代码如下:

\begin{algorithm}
\caption{IBM Model 1 with EM Algorithm}
    \begin{algorithmic}
        \State {initialize $t(c|e)=\tfrac{1}{|\mathcal{C}|}$}
        \Repeat
        \State { $count(c|e) = 0.0$}
        \State { $total(e) = 0.0$}
        \ForAll { sentence pair $\langle E_k, C_k\rangle$}
            \ForAll { words $c$ in $C_k$}
                \State { $total\_sum(c) = 0.0 $}
                \ForAll { words $e$ in $E_k$}
                    \State { $total\_sum(c)$ += $t(c|e)$}
                \EndFor
            \EndFor
            \ForAll { words $c$ in $C_k$}
                \ForAll { words $e$ in $E_k$}
                    \State { $count(c|e)$ += $t(c|e)/total\_sum(c)$}
                    \State { $total(e)$ += $t(c|e)/total\_sum(c)$}
                \EndFor
            \EndFor
        \EndFor
        \ForAll { $e$ in domain $total(\cdot)$}
            \ForAll { $c$ in domain $count(\cdot|e)$}
                \State { $t(c|e) = count(c|e)/total(e)$}
            \EndFor
        \EndFor
        \Until { $t(c|e)$ convergence}
    \end{algorithmic}
\end{algorithm}

\section{IBM Model运用}
\subsection{Alignment与Decoding~[\ref{note2}]}
\hspace{1.6em} IBM Model 1最终得到的是翻译概率$t(c|e)$,那么通过这一翻译概率表,得到最可能的Alignment,即求
\begin{equation}\label{argmaxP}
\arg\max\limits_{a} P(C,a|E)
\end{equation}
\hspace{1.6em} 而对于翻译任务来说,则是计算句子之间的对应概率
\begin{equation}\label{tranlation1}
P(C|E) = \sum_{a}P(C,a|E)
\end{equation}
\hspace{1.6em} 那么在给定一个语言模型$P(E)$的条件下,我们可以翻译中文句子$C$为$E$,其中$E$是翻译概率最高的。
\begin{equation}\label{tranlation2}
\arg\max\limits_{E} P(E)P(C|E)
\end{equation}
\subsection{Alignment with Agreement}
\hspace{1.6em} 直接使用IBM Model 1的$t(c|e)$翻译概率表,生成最可能的对齐~[\ref{note6}]
\begin{algorithm}
\caption{IBM Model 1 Alignment}\label{alignment1}
    \begin{algorithmic}
        \ForAll { sentence pair $\langle E_k, C_k\rangle$}
            \ForAll { words $e$ in $E_k$}
                \State { $bestprob = 0$}
                \ForAll { words $c$ in $C_k$}
                    \If { $t(c|e) > bestprob$ }
                        \State { $bestprob = t(c|e)$}
                    \EndIf
                \EndFor
                \ForAll { words $c$ in $C_k$}
                    \If { $t(c|e) ==  bestprob$ }
                        \State { align($k,\,e,\,c$)}
                    \EndIf
                \EndFor
            \EndFor
        \EndFor
    \end{algorithmic}
\end{algorithm}
\newline
\hspace*{1.6em} 为了进一步提高对齐的准确性,[\ref{note5}]中提出了“一致性对齐”的方式。这里,仅仅是简单应用了这一思想,通过对偶化,求得以中文翻译为英文的条件概率表$t(e|c)$,在每次比较最大概率时,从简单的$t(c|e)$转变为$(t(c|e)+t(e|c))$,利用了更多的信息,使对齐的准确率提高。

\section{运行结果分析}
\hspace{1.6em} 实验过程中发现迭代次数无需太多,$t(c|e)$之间的差距已经较为明显,本实验中迭代了100次,某些结果例如“P(说|~said)”已经十分接近1了。评价方式采用传统的Precision和Recall方式对Alignment结果进行分析,并且为了使得Precision和Recall尽量接近,对~[Alg\ref{alignment1}]~中的$bestprob$进行了下限限制,避免生成过多的alignment结果,修改后的算法如下:
\begin{algorithm}
\caption{IBM Model 1 Alignment with Agreement}\label{alignment2}
    \begin{algorithmic}
        \ForAll { sentence pair $\langle E_k, C_k\rangle$}
            \ForAll { words $e$ in $E_k$}
                \State { $bestprob = 0$}
                \ForAll { words $c$ in $C_k$}
                    \If { $(t(c|e)+t(e|c)) > bestprob$ }
                        \State { $bestprob = (t(c|e)+t(e|c))$}
                    \EndIf
                \EndFor
                \If { $bestprob>\lambda$}
                    \Comment { $\lambda$ is the threshold to constrain alignment }
                    \ForAll { words $c$ in $C_k$}
                        \If { $(t(c|e)+t(e|c)) ==  bestprob$ }
                            \State { align($k,\,e,\,c$)}
                        \EndIf
                    \EndFor
                \EndIf
            \EndFor
        \EndFor
    \end{algorithmic}
\end{algorithm}
\newline
\hspace*{1.6em} 生成的Alignment与提供的Alignment比较
\begin{center}
\begin{tabular}{|c|c|c|c|}
  \hline
  算法 & Precision & Recall & Execute Time\\ \hline
  Alg 2 & 35.18\% & 35.42\% & 7.6s\footnotemark \\ \hline
  Alg 3(Agreement) & 37.33\% & 37.68\% & 16.7s \\
  \hline
\end{tabular}
\end{center}
\footnotetext{这里的时间仅需要计算$t(c|e)$,而Alg 3中还计算了$t(e|c)$,时间上大约是两倍关系。}
\section{实验问题思考}
\hspace{1.6em} 由于实验中所给的双语语料库较小,只有各100句,所以结果并不理想,如果有更大的语料库理论上能取得更好的结果。\\
\indent 所用的双语语料库中,已经做好了中文分词的工作,减少了预处理工作,从所给文件看像是手工分词的,如果使用其他方法来分词,对齐的效果将大打折扣,分词是中文自然语言处理的第一步,也是很难的一步。此外,由于所给语料库中有标点符号、常用虚词等因素的影响,从概率上来说,这些经常出现的token可能造成了错误的对应关系,因此需要更大的语料库和更多的迭代次数来消除偏差。\\
\indent 为了支持程序的可扩展性,可以支持任意的两种语言之间的对应生成,从$t(c|e)$到获得$t(e|c)$的过程是对偶的,只需要把训练时的输入文件对调即可,代码上不需要太大的改动,为了便于阅读,把两部分代码分别写在两个文件中。\\
\indent 结合之前Michael Collins[\ref{note3}]提到的Phrased-Based Translation Model,这可以作为该项工作的前期准备,运用简单的变换,可以帮助识别出一些短语。
\section{运行说明}
\hspace{1.6em} Windows 7 64-bit, Core i3 @2.30GHz, RAM: 6GB;JDK版本:1.8.0\_60,with Eclipse Mars\\
\indent 把双语语料(test.en.txt, test.ch.txt)和给定的Alignment(test.align.txt)作为程序的输入,编译及运行方法:
\begin{itemize}
  \item \texttt{> javac IBMTest.java}
  \item \texttt{> java IBMTest}
\end{itemize}
\hspace*{1.6em} 最终生成的文件主要有$t(c|e)$表(matrixXXX.txt)、$t(e|c)$表(matrixXXXInverse.txt),XXX表示迭代的次数,以及对其结果myalignment.txt,命令行输出运行情况、对齐的准确率和迭代次数及运行时间。

\section{References}
\begin{enumerate}[{[}1{]}]
\item IBM模型学习总结 - \href{http://luowei828.blog.163.com/blog/static/31031204201123010316963/}{ http://luowei828.blog.163.com/blog/static/31031204201123010316963/ }\label{note1}
\item Philipp Koehn - \href{http://www.inf.ed.ac.uk/teaching/courses/mt/} { Machine Translation }\label{note3} Statistical Machine Translation
\item Michael Collins - \href{http://www.cs.columbia.edu/~cs4705/}{ Natural Language Processing }\label{note2} Notes on Statistical Machine Translation: IBM Models 1 and 2
\item Dark\_Scope - \href{http://blog.csdn.net/dark_scope/article/details/8774000}{NLP 学习笔记 04 (Machine Translation)}\label{note4}
\item Philipp Koehn and Adam Lopez - \href{http://www.inf.ed.ac.uk/teaching/courses/mt/syllabus.html} { Machine Translation }\label{note6}  Reading Materials on Word alignment and the expectation maximization algorithm
\item Liang P, Taskar B, Klein D. Alignment by agreement[C]//Proceedings of the main conference on Human Language Technology Conference of the North American Chapter of the Association of Computational Linguistics. Association for Computational Linguistics, 2006: 104-111.\label{note5}
\end{enumerate}

\end{document} 